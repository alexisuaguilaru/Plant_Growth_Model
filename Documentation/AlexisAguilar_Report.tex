\documentclass[12pt,a4paper]{article}

\usepackage{geometry}
\geometry{
    left=2cm, 
    right=2cm,
    top=3cm,  
    bottom=2cm
}

\usepackage[english,spanish]{babel}
\usepackage[utf8]{inputenc}
\usepackage{amsmath}

\usepackage{graphicx}
\usepackage{wrapfig}
\usepackage{makecell}
\usepackage{booktabs}

\usepackage{setspace}
\setstretch{1.5}
\setlength{\parindent}{0pt}

\usepackage{csquotes}
\usepackage{hyperref}
\usepackage[style=ieee]{biblatex}
\addbibresource{Referencias.bib}

\begin{document}
    \begin{titlepage}
        \begin{minipage}[c]{0.1\textwidth}
            \includegraphics[width=\textwidth]{./Resources/logo_unam.jpg}
        \end{minipage}
        \begin{minipage}{0.8\textwidth}
            \centering
            {\Large\textbf{Universidad Nacional Autónoma de México}\\}
            {\large\textbf{Escuela Nacional de Estudios Superiores\\\underline{Unidad Morelia}}}
        \end{minipage}
        \begin{minipage}[c]{0.1\textwidth}
            \includegraphics[width=\textwidth]{./Resources/logo_enes.jpg}
        \end{minipage}
        \vspace{3cm}

        \centering

        {\large{Proyecto Final\\}}
        {\Large\textbf{Predicción del Crecimiento Significativo en Plantas}}
        \vspace{2cm}

        {{PRESENTA:\\}}
        {\large\textbf{Alexis Uriel Aguilar Uribe}}
        \vspace{1cm} 

        {{PROFESORES:\\}}
        {\large\textbf{Dra.\ Marisol Flores Garrido}}\\
        {\large\textbf{Dr.\ Luis Miguel García Velázquez}}
        \vspace{2cm}

        {{GRADO\\}}
        {\large\textbf{Licenciatura en Tecnologías para la Información en Ciencias}}
        \vspace{2cm}

        \flushleft
        {\textbf{Número de Cuenta:\ }424060075}\\
        {\textbf{Asignatura:\ }Sistemas basados en conocimiento [Machine Learning]}
        \vspace{2cm}

        \flushright
        {\textbf{A:\ }\underline{27 de Mayo del 2025}}
        \vfill

    \end{titlepage}

    \tableofcontents
    \newpage

    \section{Introducción}
    {
        En la agricultura, como cualquier otra industria, se vuelve relevante la 
        optimización de los recursos y ganancias, es decir, reducir los insumos 
        consumidos mientras se incrementa la producción (tanto en calidad como en 
        cantidad); todo lo anterior se traduce en aplicar mejoras en diferentes 
        áreas y aspectos que convergen y se relacionan para generar ganancias y 
        reducir costos en la agricultura. Para el caso de este proyecto, el interés 
        se encuentra en el crecimiento de las plantas, bajo qué factores ambientales 
        y de cuidado propician un crecimiento significativo en las plantas.\\ 

        Para lograr el último punto, se tiene como objetivo el crear un modelo de 
        aprendizaje supervisado para la clasificación del crecimiento significativo 
        en base a los factores y mediciones relacionadas a su cuidado y ambiente.\\
    }
    \newpage

    \section{Descripción de los Datos}
    {
        El conjunto de datos que se emplearán para el proyecto se encuentra disponibles 
        en \cite{dataset_plants}, que es un conjunto de datos publicados en \href{https://www.kaggle.com/}{Kaggle} por 
        la propia comunidad. Se cuenta con siete columnas, donde seis de ellas son 
        atributos y la otra el target, referenciando a la fuente del conjunto de datos, 
        se tienen los siguientes atributos junto con su descripción y tipo de dato: 

        \begin{itemize}
            \item \textbf{Soil\_Type       } [\emph{String}]: El tipo o composición del suelo 
            en el que las plantas están creciendo o se plantan.
            
            \item \textbf{Sunlight\_Hours  } [\emph{Float}]: La duración o intensidad de la luz 
            solar que las plantas reciben.
            
            \item \textbf{Water\_Frequency } [\emph{String}]: Qué tan seguido se riegan las 
            plantas, se indica la frecuencia del riego.
            
            \item \textbf{Fertilizer\_Type } [\emph{String}]: El tipo de fertilizante usado 
            para nutrir a las plantas.
            
            \item \textbf{Temperature      } [\emph{Float}]: Las condiciones de la temperatura 
            ambiental bajo las cuales las plantas están creciendo.

            \item \textbf{Humidity         } [\emph{Float}]: El nivel de humedad en el ambiente 
            alrededor de las plantas.

            \item \textbf{Growth\_Milestone} [\emph{Integer, Target}]: Descripción o marcadores 
            que indican la etapa o eventos significativos en el proceso de crecimiento de 
            las plantas.
        \end{itemize}

        Por último, el conjunto de datos consta de $193$ instancias (filas), las diferentes 
        instancias lucen de la siguiente manera:

        \begin{center}    
            \begin{tabular}{lrll}
                \toprule
                Soil\_Type & Sunlight\_Hours & Water\_Frequency & Fertilizer\_Type \\
                \midrule
                sandy & 9.228 & daily     & none     \\
                sandy & 9.774 & weekly    & chemical \\
                clay  & 7.392 & bi-weekly & none     \\
                clay  & 6.462 & bi-weekly & organic  \\
                clay  & 8.846 & weekly    & organic  \\
                loam  & 5.985 & bi-weekly & chemical \\
                \bottomrule
            \end{tabular}
            \begin{tabular}{rrr}
                \toprule
                Temperature & Humidity & Growth\_Milestone \\
                \midrule
                33.804 & 32.815 & 0 \\
                32.549 & 61.377 & 1 \\
                31.100 & 68.600 & 0 \\
                27.517 & 34.175 & 1 \\
                27.700 & 56.800 & 1 \\
                29.757 & 57.476 & 0 \\
                \bottomrule
            \end{tabular}
        \end{center}

    }
    \newpage

    \section{Análisis Exploratorio de Datos}
    {}
    \newpage

    \section{Metodología del Proyecto}
    {}
    \newpage

    \section{Experimentos y Discusión de Resultados}
    {}
    \newpage

    \section{Análisis de los Resultados}
    {}
    \newpage

    \section{Conclusiones}
    {}
    \newpage

    \printbibliography[heading=bibintoc,title={Referencias Bibliográficas}]

\end{document}