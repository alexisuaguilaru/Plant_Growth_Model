\documentclass[12pt,a4paper]{article}

\usepackage{geometry}
\geometry{
    left=2cm, 
    right=2cm,
    top=3cm,  
    bottom=2cm
}

\usepackage[english,spanish]{babel}
\usepackage[utf8]{inputenc}
\usepackage{amsmath}

\usepackage{graphicx}
\usepackage{wrapfig}

\usepackage{setspace}
\setstretch{1.5}
\setlength{\parindent}{0pt}

\usepackage{csquotes}
\usepackage{hyperref}
\usepackage[style=ieee]{biblatex}
\addbibresource{referencias.bib}

\begin{document}
    \begin{titlepage}
        \begin{minipage}[c]{0.1\textwidth}
            \includegraphics[width=\textwidth]{./Resources/logo_unam.jpg}
        \end{minipage}
        \begin{minipage}{0.8\textwidth}
            \centering
            {\Large\textbf{Universidad Nacional Autónoma de México}\\}
            {\large\textbf{Escuela Nacional de Estudios Superiores\\\underline{Unidad Morelia}}}
        \end{minipage}
        \begin{minipage}[c]{0.1\textwidth}
            \includegraphics[width=\textwidth]{./Resources/logo_enes.jpg}
        \end{minipage}
        \vspace{3cm}

        \centering

        {\large{Proyecto Final\\}}
        {\Large\textbf{Predicción del Crecimiento Significativo en Plantas}}
        \vspace{2cm}

        {{PRESENTA:\\}}
        {\large\textbf{Alexis Uriel Aguilar Uribe}}
        \vspace{1cm} 

        {{PROFESORES:\\}}
        {\large\textbf{Dra.\ Marisol Flores Garrido}}\\
        {\large\textbf{Dr.\ Luis Miguel García Velázquez}}
        \vspace{2cm}

        {{GRADO\\}}
        {\large\textbf{Licenciatura en Tecnologías para la Información en Ciencias}}
        \vspace{2cm}

        \flushleft
        {\textbf{Número de Cuenta:\ }424060075}\\
        {\textbf{Asignatura:\ }Sistemas basados en conocimiento [Machine Learning]}
        \vspace{2cm}

        \flushright
        {\textbf{A:\ }\underline{27 de Mayo del 2025}}
        \vfill

    \end{titlepage}

    \tableofcontents
    \newpage

    \section{Introducción}
    {
        En la agricultura, como cualquier otra industria, se vuelve relevante la 
        optimización de los recursos y ganancias, es decir, reducir los insumos 
        consumidos mientras se incrementa la producción (tanto en calidad como en 
        cantidad); todo lo anterior se traduce en aplicar mejoras en diferentes 
        áreas y aspectos que convergen y se relacionan para generar ganancias y 
        reducir costos en la agricultura. Para el caso de este proyecto, el interés 
        se encuentra en el crecimiento de las plantas, bajo qué factores ambientales 
        y de cuidado propician un crecimiento significativo en las plantas.\\ 

        Para lograr el último punto, se tiene como objetivo el crear un modelo de 
        aprendizaje supervisado para la clasificación del crecimiento significativo 
        en base a los factores y mediciones relacionadas a su cuidado y ambiente.\\
    }
    \newpage

    \section{Descripción de los Datos}
    {}
    \newpage

    \section{Análisis Exploratorio de Datos}
    {}
    \newpage

    \section{Metodología del Proyecto}
    {}
    \newpage

    \section{Experimentos y Discusión de Resultados}
    {}
    \newpage

    \section{Análisis de los Resultados}
    {}
    \newpage

    \section{Conclusiones}
    {}
    \newpage

    \printbibliography[heading=bibintoc,title={Referencias Bibliográficas}]

\end{document}